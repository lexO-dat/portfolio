\documentclass[a4paper,11pt]{article}
\usepackage[utf8]{inputenc}
\usepackage[spanish]{babel}
\usepackage{enumitem}
\usepackage{geometry}
\usepackage{parskip}
\usepackage{hyperref}
\geometry{top=2cm, bottom=2cm, left=2cm, right=2cm}

% Datos de contacto en la cabecera
\begin{document}
\small % Cambia a \footnotesize para una fuente más pequeña, o a \scriptsize para aún más pequeño
\begin{center}
    {\LARGE \textbf{LUCAS ABELLO}}\\
    \vspace{0.2cm}
    Computer Science Student | AI \& Software Engineering Enthusiast\\
    \vspace{0.2cm}
    +56973756474 \textbullet{} \href{mailto:lucas.abello@mail.udp.cl}{lucas.abello@mail.udp.cl} \textbullet{} LinkedIn: \href{https://www.linkedin.com/in/lucasabello/}{lucasabello} \textbullet{} \href{https://github.com/lexO-dat}{GitHub: lexO-dat}
\end{center}

\section*{About Me}
I am a computer science student specializing in AI applications with hands-on experience in multi-agent systems, LLM fine-tuning, NLP, and RAG implementations. I have experience developing hierarchical ML architectures and software-hardware integrations. Active competitor in IEEE Xtreme and hackathons, demonstrating problem-solving and collaborative skills. Seeking to advance AI-driven software development with real-world impact while continuously expanding my technical expertise.
\\
\rule{\linewidth}{0.1pt}

\section*{Projects}

\textbf{pparser} \href{https://github.com/lexO-dat/pparser}{github.com/pparser} \hfill Present \\
\textit{ Multi-agent PDF-to-Markdown conversion system that transforms documents into LLM ready formats for RAG systems and other applications.} 
\begin{itemize}
    \item Used langGraph for development of the LLM system and created 7 different agents that work alongside to accomplish the task.
    \item Created the system as a Python package to easily import it and use it on any python code or even from the terminal.
    \item Technologies used: Python, LangGraph and LLMs.
\end{itemize}

\textbf{CELLM} \href{https://github.com/lexO-dat/CELLM}{github.com/CELLM} \hfill 2024-2025 \\
\textit{Developed an automated genetic circuit design system during my first research internship that generates biological circuits based on logical descriptions provided by the user.}
\begin{itemize}
\item Developed a multi-agent system for automated genetic circuit construction
\item Fine-tuned LLM agents to analyze and synthesize biological circuits
\item Implemented RAG (Retrieval-Augmented Generation) system to extract biological components, accelerating configuration file selection
\item Built Golang CLI integration for the pipeline execution
\item Technologies used: Python, Golang
\end{itemize}

\textbf{Deaf-PI} \href{https://github.com/lexO-dat/SLML}{github.com/SLML} \hfill 2024 \\
\textit{Sign language translation device}
\begin{itemize}
    \item Developed hierarchical machine learning architecture with cluster-based sign detection: primary models identify sign clusters, followed by specialized models for precise sign recognition within each cluster
    \item Integrated custom API for seamless communication with Raspberry Pi hardware platform
    \item Built custom ML models using Python, TensorFlow, and OpenCV for real-time image processing and sign detection
\end{itemize}
\\
\hline
\section*{Education}
\textbf{Universidad Diego Portales} \hfill Santiago de Chile \\
Civil Engineer in Computer Science and Telecommunications \hfill 03/2022 - Present
\\
\hline
\section*{Languages}
\textbf{English} \hfill B2

\end{document}
